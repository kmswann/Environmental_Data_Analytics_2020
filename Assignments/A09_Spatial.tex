% Options for packages loaded elsewhere
\PassOptionsToPackage{unicode}{hyperref}
\PassOptionsToPackage{hyphens}{url}
%
\documentclass[
]{article}
\usepackage{lmodern}
\usepackage{amssymb,amsmath}
\usepackage{ifxetex,ifluatex}
\ifnum 0\ifxetex 1\fi\ifluatex 1\fi=0 % if pdftex
  \usepackage[T1]{fontenc}
  \usepackage[utf8]{inputenc}
  \usepackage{textcomp} % provide euro and other symbols
\else % if luatex or xetex
  \usepackage{unicode-math}
  \defaultfontfeatures{Scale=MatchLowercase}
  \defaultfontfeatures[\rmfamily]{Ligatures=TeX,Scale=1}
\fi
% Use upquote if available, for straight quotes in verbatim environments
\IfFileExists{upquote.sty}{\usepackage{upquote}}{}
\IfFileExists{microtype.sty}{% use microtype if available
  \usepackage[]{microtype}
  \UseMicrotypeSet[protrusion]{basicmath} % disable protrusion for tt fonts
}{}
\makeatletter
\@ifundefined{KOMAClassName}{% if non-KOMA class
  \IfFileExists{parskip.sty}{%
    \usepackage{parskip}
  }{% else
    \setlength{\parindent}{0pt}
    \setlength{\parskip}{6pt plus 2pt minus 1pt}}
}{% if KOMA class
  \KOMAoptions{parskip=half}}
\makeatother
\usepackage{xcolor}
\IfFileExists{xurl.sty}{\usepackage{xurl}}{} % add URL line breaks if available
\IfFileExists{bookmark.sty}{\usepackage{bookmark}}{\usepackage{hyperref}}
\hypersetup{
  pdftitle={Assignment 9: Spatial Analysis},
  pdfauthor={Kristine Swann},
  hidelinks,
  pdfcreator={LaTeX via pandoc}}
\urlstyle{same} % disable monospaced font for URLs
\usepackage[margin=2.54cm]{geometry}
\usepackage{graphicx,grffile}
\makeatletter
\def\maxwidth{\ifdim\Gin@nat@width>\linewidth\linewidth\else\Gin@nat@width\fi}
\def\maxheight{\ifdim\Gin@nat@height>\textheight\textheight\else\Gin@nat@height\fi}
\makeatother
% Scale images if necessary, so that they will not overflow the page
% margins by default, and it is still possible to overwrite the defaults
% using explicit options in \includegraphics[width, height, ...]{}
\setkeys{Gin}{width=\maxwidth,height=\maxheight,keepaspectratio}
% Set default figure placement to htbp
\makeatletter
\def\fps@figure{htbp}
\makeatother
\setlength{\emergencystretch}{3em} % prevent overfull lines
\providecommand{\tightlist}{%
  \setlength{\itemsep}{0pt}\setlength{\parskip}{0pt}}
\setcounter{secnumdepth}{-\maxdimen} % remove section numbering

\title{Assignment 9: Spatial Analysis}
\author{Kristine Swann}
\date{}

\begin{document}
\maketitle

\hypertarget{overview}{%
\subsection{OVERVIEW}\label{overview}}

This exercise accompanies the lessons in Environmental Data Analytics on
spatial analysis.

\hypertarget{directions}{%
\subsection{Directions}\label{directions}}

\begin{enumerate}
\def\labelenumi{\arabic{enumi}.}
\tightlist
\item
  Use this document to create code for a map. You will \textbf{NOT} be
  turning in the knitted Rmd file this time, only the pdf output for a
  map.
\item
  When you have produced your output, submit \textbf{only} the pdf file
  for the map, without any code. Please name your file
  ``StudentName\_A09\_Spatial.pdf''.
\end{enumerate}

The completed exercise is due on Thursday, March 19 at 1:00 pm.

\hypertarget{create-a-map}{%
\subsection{Create a map}\label{create-a-map}}

You have three options for this assignment, and you will turn in just
\textbf{one} final product. Feel free to choose the option that will be
most beneficial to you. For all options, to earn full points you should
use best practices for data visualization that we have covered in
previous assignments (e.g., relabeling axes and legends, choosing
non-default color palettes, etc.).

Here are your three options:

\begin{enumerate}
\def\labelenumi{\arabic{enumi}.}
\item
  Reproduce figure 1b from the spatial lesson, found in section 3.2.2.
  You may choose a state other than North Carolina, but your map should
  still contain the spatial features contained in figure 1b in the
  ``img'' folder.
\item
  Create a new map that mixes spatial and tabular data, as in section
  3.3 of the spatial lesson. You may use the maps created in the lesson
  as an example, but your map should contain data other than
  precipitation days per year. This map should include:
\end{enumerate}

\begin{itemize}
\tightlist
\item
  State boundary layer
\item
  Basin boundary layer
\item
  Gage layer
\item
  Tabular data (as an aesthetic for one of the layers)
\end{itemize}

\begin{enumerate}
\def\labelenumi{\arabic{enumi}.}
\setcounter{enumi}{2}
\tightlist
\item
  Create a map of any other spatial data. This could be data from the
  spatial lesson, data from our other course datasets (e.g., the Litter
  dataset includes latitude and longitude of trap sites), or another
  dataset of your choosing. Your map should include:
\end{enumerate}

\begin{itemize}
\tightlist
\item
  One or more layers with polygon features (e.g., country boundaries,
  watersheds)
\item
  One or more layers with point and/or line features (e.g., sampling
  sites, roads)
\item
  Tabular data that correpond to one of the layers, specified as an
  aesthetic (e.g., total litter biomass at each trap, land cover class
  at each trap)
\end{itemize}

Hint: One package that may come in handy here is the \texttt{maps}
package, which contains several options for basemaps that cover
political and geologic boundaries.

\end{document}
