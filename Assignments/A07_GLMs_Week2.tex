% Options for packages loaded elsewhere
\PassOptionsToPackage{unicode}{hyperref}
\PassOptionsToPackage{hyphens}{url}
%
\documentclass[
]{article}
\usepackage{lmodern}
\usepackage{amssymb,amsmath}
\usepackage{ifxetex,ifluatex}
\ifnum 0\ifxetex 1\fi\ifluatex 1\fi=0 % if pdftex
  \usepackage[T1]{fontenc}
  \usepackage[utf8]{inputenc}
  \usepackage{textcomp} % provide euro and other symbols
\else % if luatex or xetex
  \usepackage{unicode-math}
  \defaultfontfeatures{Scale=MatchLowercase}
  \defaultfontfeatures[\rmfamily]{Ligatures=TeX,Scale=1}
\fi
% Use upquote if available, for straight quotes in verbatim environments
\IfFileExists{upquote.sty}{\usepackage{upquote}}{}
\IfFileExists{microtype.sty}{% use microtype if available
  \usepackage[]{microtype}
  \UseMicrotypeSet[protrusion]{basicmath} % disable protrusion for tt fonts
}{}
\makeatletter
\@ifundefined{KOMAClassName}{% if non-KOMA class
  \IfFileExists{parskip.sty}{%
    \usepackage{parskip}
  }{% else
    \setlength{\parindent}{0pt}
    \setlength{\parskip}{6pt plus 2pt minus 1pt}}
}{% if KOMA class
  \KOMAoptions{parskip=half}}
\makeatother
\usepackage{xcolor}
\IfFileExists{xurl.sty}{\usepackage{xurl}}{} % add URL line breaks if available
\IfFileExists{bookmark.sty}{\usepackage{bookmark}}{\usepackage{hyperref}}
\hypersetup{
  pdftitle={Assignment 7: GLMs week 2 (Linear Regression and beyond)},
  pdfauthor={Kristine Swann},
  hidelinks,
  pdfcreator={LaTeX via pandoc}}
\urlstyle{same} % disable monospaced font for URLs
\usepackage[margin=2.54cm]{geometry}
\usepackage{graphicx,grffile}
\makeatletter
\def\maxwidth{\ifdim\Gin@nat@width>\linewidth\linewidth\else\Gin@nat@width\fi}
\def\maxheight{\ifdim\Gin@nat@height>\textheight\textheight\else\Gin@nat@height\fi}
\makeatother
% Scale images if necessary, so that they will not overflow the page
% margins by default, and it is still possible to overwrite the defaults
% using explicit options in \includegraphics[width, height, ...]{}
\setkeys{Gin}{width=\maxwidth,height=\maxheight,keepaspectratio}
% Set default figure placement to htbp
\makeatletter
\def\fps@figure{htbp}
\makeatother
\setlength{\emergencystretch}{3em} % prevent overfull lines
\providecommand{\tightlist}{%
  \setlength{\itemsep}{0pt}\setlength{\parskip}{0pt}}
\setcounter{secnumdepth}{-\maxdimen} % remove section numbering

\title{Assignment 7: GLMs week 2 (Linear Regression and beyond)}
\author{Kristine Swann}
\date{}

\begin{document}
\maketitle

\hypertarget{overview}{%
\subsection{OVERVIEW}\label{overview}}

This exercise accompanies the lessons in Environmental Data Analytics on
generalized linear models.

\hypertarget{directions}{%
\subsection{Directions}\label{directions}}

\begin{enumerate}
\def\labelenumi{\arabic{enumi}.}
\tightlist
\item
  Change ``Student Name'' on line 3 (above) with your name.
\item
  Work through the steps, \textbf{creating code and output} that fulfill
  each instruction.
\item
  Be sure to \textbf{answer the questions} in this assignment document.
\item
  When you have completed the assignment, \textbf{Knit} the text and
  code into a single PDF file.
\item
  After Knitting, submit the completed exercise (PDF file) to the
  dropbox in Sakai. Add your last name into the file name (e.g.,
  ``Salk\_A06\_GLMs\_Week1.Rmd'') prior to submission.
\end{enumerate}

The completed exercise is due on Tuesday, February 25 at 1:00 pm.

\hypertarget{set-up-your-session}{%
\subsection{Set up your session}\label{set-up-your-session}}

\begin{enumerate}
\def\labelenumi{\arabic{enumi}.}
\item
  Set up your session. Check your working directory, load the tidyverse,
  nlme, and piecewiseSEM packages, import the \emph{raw} NTL-LTER raw
  data file for chemistry/physics, and import the processed litter
  dataset. You will not work with dates, so no need to format your date
  columns this time.
\item
  Build a ggplot theme and set it as your default theme.
\end{enumerate}

\begin{verbatim}
## [1] "C:/Users/krist/Box Sync/Spring 2020/R/Environmental_Data_Analytics_2020"
\end{verbatim}

\hypertarget{ntl-lter-test}{%
\subsection{NTL-LTER test}\label{ntl-lter-test}}

Research question: What is the best set of predictors for lake
temperatures in July across the monitoring period at the North Temperate
Lakes LTER?

\begin{enumerate}
\def\labelenumi{\arabic{enumi}.}
\setcounter{enumi}{2}
\tightlist
\item
  Wrangle your NTL-LTER dataset with a pipe function so that it contains
  only the following criteria:
\end{enumerate}

\begin{itemize}
\tightlist
\item
  Only dates in July (hint: use the daynum column). No need to consider
  leap years.
\item
  Only the columns: lakename, year4, daynum, depth, temperature\_C
\item
  Only complete cases (i.e., remove NAs)
\end{itemize}

\begin{enumerate}
\def\labelenumi{\arabic{enumi}.}
\setcounter{enumi}{3}
\tightlist
\item
  Run an AIC to determine what set of explanatory variables (year4,
  daynum, depth) is best suited to predict temperature. Run a multiple
  regression on the recommended set of variables.
\end{enumerate}

\begin{verbatim}
## Start:  AIC=24461.34
## temperature_C ~ lakename + year4 + daynum + depth
## 
##            Df Sum of Sq    RSS   AIC
## <none>                  120062 24461
## - year4     1       184 120245 24474
## - daynum    1      1346 121407 24568
## - lakename  8     21056 141118 26016
## - depth     1    403139 523201 38770
\end{verbatim}

\begin{verbatim}
## 
## Call:
## lm(formula = temperature_C ~ lakename + year4 + daynum + depth, 
##     data = cp)
## 
## Coefficients:
##              (Intercept)     lakenameCrampton Lake    lakenameEast Long Lake  
##                 45.17306                   4.71362                  -1.46041  
## lakenameHummingbird Lake         lakenamePaul Lake        lakenamePeter Lake  
##                 -4.73042                   0.99422                   1.44048  
##     lakenameTuesday Lake         lakenameWard Lake    lakenameWest Long Lake  
##                 -1.38445                  -0.46590                  -0.16847  
##                    year4                    daynum                     depth  
##                 -0.01588                   0.04157                  -1.96540
\end{verbatim}

\begin{verbatim}
## 
## Call:
## lm(formula = temperature_C ~ lakename + year4 + daynum + depth, 
##     data = cp)
## 
## Residuals:
##     Min      1Q  Median      3Q     Max 
## -7.8938 -3.0274 -0.2114  2.7781 15.2926 
## 
## Coefficients:
##                           Estimate Std. Error  t value Pr(>|t|)    
## (Intercept)              45.173063   8.248578    5.476 4.45e-08 ***
## lakenameCrampton Lake     4.713617   0.382185   12.333  < 2e-16 ***
## lakenameEast Long Lake   -1.460406   0.343271   -4.254 2.12e-05 ***
## lakenameHummingbird Lake -4.730421   0.459795  -10.288  < 2e-16 ***
## lakenamePaul Lake         0.994222   0.331643    2.998 0.002726 ** 
## lakenamePeter Lake        1.440479   0.331406    4.347 1.40e-05 ***
## lakenameTuesday Lake     -1.384450   0.336476   -4.115 3.91e-05 ***
## lakenameWard Lake        -0.465900   0.464619   -1.003 0.316003    
## lakenameWest Long Lake   -0.168474   0.341961   -0.493 0.622257    
## year4                    -0.015885   0.004118   -3.857 0.000115 ***
## daynum                    0.041574   0.003985   10.432  < 2e-16 ***
## depth                    -1.965403   0.010885 -180.566  < 2e-16 ***
## ---
## Signif. codes:  0 '***' 0.001 '**' 0.01 '*' 0.05 '.' 0.1 ' ' 1
## 
## Residual standard error: 3.516 on 9710 degrees of freedom
## Multiple R-squared:  0.7803, Adjusted R-squared:   0.78 
## F-statistic:  3135 on 11 and 9710 DF,  p-value: < 2.2e-16
\end{verbatim}

\begin{enumerate}
\def\labelenumi{\arabic{enumi}.}
\setcounter{enumi}{4}
\tightlist
\item
  What is the final set of explanatory variables that predict
  temperature from your multiple regression? How much of the observed
  variance does this model explain?
\end{enumerate}

\begin{quote}
Answer: Wow. The model explains 78\% of the variance. That's crazy high!
Final set of explanatory variables that predict temp: lakes (crampton
lake, east long lake, hummingbird lake, paul lake, peter lake, tuesday
lake, ward lake, and long lake), year, day within July, and depth. The
only lake that didn't have a specific coefficient listed was Central
Long Lake.
\end{quote}

\begin{enumerate}
\def\labelenumi{\arabic{enumi}.}
\setcounter{enumi}{5}
\tightlist
\item
  Run an interaction effects ANCOVA to predict temperature based on
  depth and lakename from the same wrangled dataset.
\end{enumerate}

\begin{verbatim}
## 
## Call:
## lm(formula = temperature_C ~ lakename * depth, data = cp)
## 
## Residuals:
##     Min      1Q  Median      3Q     Max 
## -7.6455 -2.9133 -0.2879  2.7567 16.3606 
## 
## Coefficients:
##                                Estimate Std. Error t value Pr(>|t|)    
## (Intercept)                     22.9455     0.5861  39.147  < 2e-16 ***
## lakenameCrampton Lake            2.2173     0.6804   3.259  0.00112 ** 
## lakenameEast Long Lake          -4.3884     0.6191  -7.089 1.45e-12 ***
## lakenameHummingbird Lake        -2.4126     0.8379  -2.879  0.00399 ** 
## lakenamePaul Lake                0.6105     0.5983   1.020  0.30754    
## lakenamePeter Lake               0.2998     0.5970   0.502  0.61552    
## lakenameTuesday Lake            -2.8932     0.6060  -4.774 1.83e-06 ***
## lakenameWard Lake                2.4180     0.8434   2.867  0.00415 ** 
## lakenameWest Long Lake          -2.4663     0.6168  -3.999 6.42e-05 ***
## depth                           -2.5820     0.2411 -10.711  < 2e-16 ***
## lakenameCrampton Lake:depth      0.8058     0.2465   3.268  0.00109 ** 
## lakenameEast Long Lake:depth     0.9465     0.2433   3.891  0.00010 ***
## lakenameHummingbird Lake:depth  -0.6026     0.2919  -2.064  0.03903 *  
## lakenamePaul Lake:depth          0.4022     0.2421   1.662  0.09664 .  
## lakenamePeter Lake:depth         0.5799     0.2418   2.398  0.01649 *  
## lakenameTuesday Lake:depth       0.6605     0.2426   2.723  0.00648 ** 
## lakenameWard Lake:depth         -0.6930     0.2862  -2.421  0.01548 *  
## lakenameWest Long Lake:depth     0.8154     0.2431   3.354  0.00080 ***
## ---
## Signif. codes:  0 '***' 0.001 '**' 0.01 '*' 0.05 '.' 0.1 ' ' 1
## 
## Residual standard error: 3.471 on 9704 degrees of freedom
## Multiple R-squared:  0.7861, Adjusted R-squared:  0.7857 
## F-statistic:  2097 on 17 and 9704 DF,  p-value: < 2.2e-16
\end{verbatim}

\begin{enumerate}
\def\labelenumi{\arabic{enumi}.}
\setcounter{enumi}{6}
\tightlist
\item
  Is there a significant interaction between depth and lakename? How
  much variance in the temperature observations does this explain?
\end{enumerate}

\begin{quote}
Answer: There is a significant interaction between depth and lakename,
with an overall model p-value \textless{} 0.0001. The adjusted R2 value
is 0.79, which is even higher than the other lm! The interactions
between depth and individual lakes have varied p-values, with Hummingird
being the only one without significant interaction with depth (p-values
\textgreater{} 0.05). I suppose all this suggests that site specific
conditions include total depth, so there's an interaction.
\end{quote}

\begin{enumerate}
\def\labelenumi{\arabic{enumi}.}
\setcounter{enumi}{7}
\tightlist
\item
  Create a graph that depicts temperature by depth, with a separate
  color for each lake. Add a geom\_smooth (method = ``lm'', se = FALSE)
  for each lake. Make your points 50 \% transparent. Adjust your y axis
  limits to go from 0 to 35 degrees. Clean up your graph to make it
  pretty.
\end{enumerate}

\includegraphics{A07_GLMs_Week2_files/figure-latex/unnamed-chunk-4-1.pdf}

\begin{enumerate}
\def\labelenumi{\arabic{enumi}.}
\setcounter{enumi}{8}
\tightlist
\item
  Run a mixed effects model to predict dry mass of litter. We already
  know that nlcdClass and functionalGroup have a significant
  interaction, so we will specify those two variables as fixed effects
  with an interaction. We also know that litter mass varies across plot
  ID, but we are less interested in the actual effect of the plot itself
  but rather in accounting for the variance among plots. Plot ID will be
  our random effect.
\end{enumerate}

\begin{enumerate}
\def\labelenumi{\alph{enumi}.}
\tightlist
\item
  Build and run a mixed effects model.
\item
  Check the difference between the marginal and conditional R2 of the
  model.
\end{enumerate}

\begin{verbatim}
##   Response   family     link method  Marginal Conditional
## 1  dryMass gaussian identity   none 0.2465822   0.2679023
\end{verbatim}

\begin{enumerate}
\def\labelenumi{\alph{enumi}.}
\setcounter{enumi}{1}
\tightlist
\item
  continued\ldots{} How much more variance is explained by adding the
  random effect to the model?
\end{enumerate}

\begin{quote}
Answer: approximately 2 \%, with marginal r2 = 0.24 and conditional =
0.26.
\end{quote}

\begin{enumerate}
\def\labelenumi{\alph{enumi}.}
\setcounter{enumi}{2}
\tightlist
\item
  Run the same model without the random effect.
\item
  Run an anova on the two tests.
\end{enumerate}

\begin{verbatim}
##   Response   family     link method R.squared
## 1  dryMass gaussian identity   none 0.2515836
\end{verbatim}

\begin{verbatim}
##              Model df      AIC      BIC    logLik   Test  L.Ratio p-value
## litter.mixed     1 26 9038.575 9179.479 -4493.287                        
## litter.fixed     2 25 9058.088 9193.573 -4504.044 1 vs 2 21.51338  <.0001
\end{verbatim}

\begin{enumerate}
\def\labelenumi{\alph{enumi}.}
\setcounter{enumi}{3}
\tightlist
\item
  continued\ldots{} Is the mixed effects model a better model than the
  fixed effects model? How do you know?
\end{enumerate}

\begin{quote}
Answer: The fixed effect is slightly better than the mixed effect
because the AIC is slightly lower (mixed = 9038.6; fixed = 9038.1).
\end{quote}

\end{document}
